%%% préamble pour les graphiques %%%
\usepackage{graphicx,color} 
% \usepackage{xcolor}
\usepackage[svgnames]{xcolor} %seuls les noms des couleurs sont pris dans SVG
%%% pour des graphiques incrustés dans le texte, déconseillé pour une thèse,
%\usepackage{wrapfig,picins} %  %éviter wrapfig et picins au profit de floatflt
%\usepackage{floatflt}
%%% Pour un meilleur contrôle des objets flottants (figures, tables)
\renewcommand{\topfraction}{0.7}     % autorise 70% page de graphique en haut
\renewcommand{\bottomfraction}{0.5}  % autorise 50% page de graphique en bas
\renewcommand{\floatpagefraction}{0.7}
\renewcommand{\textfraction}{0.1}
%%% Format français des légendes
\usepackage{subcaption}
% \captionsetup[figure]{name=Fig.,labelsep=quad,labelfont=normalfont,textfont=sl,%
   % singlelinecheck=true,width=0.9\linewidth}
%%% Autres paquets
%\usepackage{placeins}   % pour  \FloatBarrier
%\usepackage{epstopdf}   % pour inclure des eps
%\usepackage{pdfpages}   % por inclure de portions d fichiers PDF
%\usepackage{float} \usepackage{nonfloat} \usepackage{endfloat} \usepackage{topfloat}
%%% Chemin des figures
%\graphicspath{{./fig1/},{../fig2/}}

